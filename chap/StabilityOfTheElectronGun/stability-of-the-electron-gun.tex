\chapter{Current and Needed Stability of the Electron Gun}
In this chapter the current stability of the RF system powering the electron gun is analyzed and the needed stability is defined.

\section{Metrics to quantify stability}\label{sec:metrics}
``Stability'' can have different meanings depending on the context. In case of signal processing, a signal is usually said to be \textit{stable} if it has only little variation around its mean or some target value, i.e. the mean has to be constant and the variance stays below some threshold. 
Stability is not to be confused with stationarity, which requires the mean and the variance and the autocorrelation stay constant over time \cite{Guthrie2020}. To express stability as a single numerical value, there are several possibilities, some are described in the following.

\paragraph{Relative Standard Deviation}
This measures the stability as the standard deviation but related to the mean value to make it comparable to other quantities with different scaling or units.

The relative, or percentual, standard deviation of $x$ is defined using the mean $\mu_x$ and the standard deviation $\sigma_x$ as
\begin{equation}
\%STD_x := \frac{\sigma_x}{\mu_x}
\end{equation}

\paragraph{Mean Squared Error}
The mean squared error sums up the squared errors $\left(x[n] - x_t[n]\right)^2$ of $x[n]$ from a set value $x_t[n]$. To remove the effect of the length of the data sequence, the sum is devided by the length of the sequence $N$:
\begin{equation}
MSE_x := \frac{1}{N} \sum_{n=0}^{N-1} \left(x[n] - x_t[n]\right)^2
\end{equation}

\paragraph{Relative Power of Most Prominent Noise}
This novel approach compares the power of the most prominent noise power source $P_{noise,\,max}$ of the signal $x$ with the total power $P_x$:
\begin{equation}
MPN_x := \frac{P_{noise,\,max}}{P_x}
\end{equation}

\section{Determine the Needed Stability for FLUTE Operation}


\section{Analyzing the Current Stability}



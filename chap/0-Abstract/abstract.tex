\section*{Abstract}
The \textbf{F}erninfrarot \textbf{L}inac- \textbf{U}nd \textbf{T}est-\textbf{E}xperiment (FLUTE), a compact linear accelerator, is currently designed and under commission at the Karlsruhe Institute of Technology (KIT). Its main purposes are to serve as a technology platform for accelerator research, the generation of strong and ultra short THz pulses and in the future as an injection device for \textbf{c}ompact \textbf{St}orage ring for \textbf{A}ccelerator \textbf{R}esearch and \textbf{T}echnology (cSTART).

At the current commissioning state, the klystron which powers the electron gun/RF cavity and in later stages the linear accelerator is fed by a pulse forming network, which is driven by a high voltage source connected to mains power. For high and a stable output power of the cavity resonator, several parameters have to be tuned to the correct values and kept inside of sometimes small tolerance bands.\\
In the past, the coolant temperature of the cavities water cooling system and the dependency of the pulse forming network output of the mains voltage phase were predominant sources of instability. After dealing with these issues, the cavity output power stability was improved significantly but further improvements to the stability were still desired.

In this work instead of passively optimizing the stability of system parameters, an active approach is evaluated. By controlling the amplitude of the RF input signal of klystron, which is easily possible since it is low power, the effects of noise and/or drifts are mitigated.\\
Here it is evaluated if a simple of the shelf voltage controllable attenuator is a feasible choice to control the RF input signal, which input data should be used and which algorithm and/or control system is suitable to determine the needed attenuator setting to stabilize RF output (of the cavity).\\
Furthermore since the next stage in the system depends on a stable electron bunch charge rather than cavity power, it is determined whether the charge measurements of a Faraday cup can be used to directly control electron bunch charge.

\section*{Kurzfassung}
--TODO--
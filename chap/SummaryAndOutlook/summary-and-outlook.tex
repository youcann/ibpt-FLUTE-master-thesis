\chapter{Summary and Outlook}
For the operational phase of \gls{flute}, a stable electron energy enabling stable \gls{thz} energies will be vital to perform scientific experiments.

At the start of the thesis, the stability of the electron gun was unsatisfactory. As the main source of the instability, the cooling system of the electron gun's body was identified.

This thesis shows that with a control system that interacts with the low power input signal of the klystron, it is possible to improve the stability by creating a closed-loop feedback system using readouts from the \gls{epics} control system as sensors and a controllable \gls{rf} attenuator in the signal path as actuator. The system transfer function (plant) between the attenuator and the power in the electron gun cavity shows $PT_2$ behavior, so control with a \gls{pid} controller is possible and its positive effects on the stability are shown.

However the studies here also show the limitations of a traditional \gls{lti} control system. The controller should be designed with a fast response time and sufficient integral gain to reject disturbances quickly. This leads, however, to small safety margins in the gain and phase response which makes it likely for the control system to become unstable if system parameters change slightly. Also, the parameterization of the necessary measurement filter is a trade-off between stability of the control system and measurement noise rejection. The control system could be further optimized by switching to a totally different controller architecture. A possible choice might be model predictive control, which exploits the already identified plant transfer function but determines the output through an optimization process rather than a linear system.

At the moment of writing (June 2021), the \gls{linac} section is under commission, so \gls{flute} is shut down for several weeks. With the \gls{linac} in place and the low energy section operational again, measurements of the electron energies before and after the \gls{linac} can be performed. The method currently used is based on the electron bunch is steered with an electromagnet by the Lorentz force (see \autoref{eq:fl}). By adjusting the electromagnet's coil current in such a way that the electron beam hits a camera screen in the center, the electron energy can be calculated from the coil current. With this destructive and slow process it will be possible to show if the stabilization of the cavity \gls{rf} power actually improves the energies of the accelerated electrons and if the approach using the Faraday cup is an improvement over the cavity \gls{rf} power solution.

In the near future, both the electron gun and the klystron are to be upgraded to new versions. Switching to another gun and/or a different \gls{rf} system could lead to an entirely different behavior of the whole system, but the techniques described in this thesis are universally applicable and the control system can easily be modified for usage with different components.
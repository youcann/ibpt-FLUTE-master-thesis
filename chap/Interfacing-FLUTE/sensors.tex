\chapter{Machine Diagnostics and Sensors}
Most signals to be useful for diagnostics are available in EPICS. Retrieving their current value is therefore straight forward.

For example using the command line tool \texttt{caget} the last value of a PV can be obtained independent of the programming language used, however a syscall is needed which comes with a (slight) overhead and is inconvenient to use.

At least for Python and C++ there are EPICS libraries directly providing useful EPICS commands. A Python example of how to get the current cavity power is shown in \autoref{lst:sensors-epicscaget}.

\begin{lstlisting}[style=python,caption = Get an EPICS PV with python, label = lst:sensors-epicscaget]
from epics import caget
cavity_power=caget("F:RF:LLRF:01:GunCav1:Power:Out")
\end{lstlisting}

\section{Klystron and Cavity RF power}
The RF power directly out of the klystron and right before the cavity are measured with TODO and available in EPICS.

\section{Water Temperatures}
Certain components of FLUTE that generate a lot of heat when in operation are water cooled. The cooling water temperatures are measured and delivered to EPICS. 

\section{Electron Bunch Charge}
A faraday cup FARC-04\cite{radiabeamFaradayCups} (by RadiaBeam Technologies) is used to measure the charge of an electron bunch.

At the moment it is read out with a PCB 421A25\cite{pcbsynotechPCB421A25Charge} charge amplifier. With the TODO-DAQ the output voltage signal ($\sim$ the charge) is fed into EPICS.






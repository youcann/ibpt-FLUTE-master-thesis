\chapter*{Abstract}
The compact linear accelerator \gls{flute} is currently under commission at \gls{kit}. Its main purposes are to serve as a technology platform for accelerator research and the generation of strong and ultra short \gls{thz} pulses.

The electron gun and the \gls{linac} are powered by a klystron. It is fed by a pulse forming network, which is driven by a high voltage source connected to mains power. For stable energies of the generated \gls{thz} pulses, the electron energies have to be stable. To ensure stable energies of the emitted electron bunches, several parameters of the gun, such as temperature and the \gls{rf} power supply from the klystron, have to stay inside tight tolerance bands.

In this work, instead of passively optimizing the stability of system components, such as the water coolers or power supplies, an active approach with a closed feed-back loop is evaluated. By means of a control system, the amplitude of the low power \gls{rf} input signal of the klystron is manipulated to mitigate the effects of noise and drifts on the electron energy. As there is currently no sensor to measure the electron energies of all the electron bunches, the \gls{rf} power in the first gun cavity is used instead as an estimator for the electron energy stability.

As part of the development process, first the stability issue is analyzed and metrics for quantifying the stability are defined. Then, an appropriate solution, a linear, discrete time control system, is proposed. In order to implement it, all the necessary building blocks of such a control system are treated in detail. First the necessary sensors and actuators are selected. Then the controller and the measurement filter are designed. To verify the designed system, first an offline simulation on a computer is performed which shows qualitatively a satisfactory disturbance rejection with a measured disturbance signal from \gls{flute}. 

Then the control system is implemented as an algorithm with a fixed-interval control loop using the Python programming language. A graphical user interface, written in Qml, provides the user with plots and status information and allows the fine-tuning of the controller.

The following experiments at \gls{flute} show results in accordance to the simulation. That is, the stability, when defined as the relative standard deviation, is improved greatly by about a factor of \num{25}.

Finally ways to refine the control system are regarded. First by using disturbance feed-forward of the change in waster temperature, the control system is made more robust and achieves the same results. Second the usage of a Faraday cup, which measures total electron charge provides a potentially better representation of the electron energies, however as the electron beam is lost in the cup, its usages are limited.
\cleardoublepage

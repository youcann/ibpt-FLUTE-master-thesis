\section*{Abstract}
\gls{flute}, a compact linear accelerator, is currently designed and under commission at \gls{kit}. Its main purposes are to serve as a technology platform for accelerator research, the generation of strong and ultra short \gls{thz} pulses and in the future as an injection device for the \gls{cstart}.

At the current commissioning state, the klystron which powers the microwave cavity in the electron gun and, in later stages, the linear accelerator is fed by a pulse forming network, which is driven by a high voltage source connected to mains power. To ensure stable energies of the emitted electron bunches, several parameters of the cavity, such as temperature, as well as the power supply, such as \gls{rf} power, have to stay inside tight tolerance bands.

In the past, a predominant source of instability were slow drifts of the \gls{rf} power due to interference with the \SI{50}{\hertz} of the mains voltage. After dealing with this issue, the cavity \gls{rf} power stability was improved significantly, which also improves the electron stability. But further improvements to the stability are still desired to make the whole system usable for scientific experiments, such as \gls{thz} spectroscopy.

In this work, instead of passively optimizing the stability of system components, an active approach is evaluated. By means of a control system, the amplitude of the low power \gls{rf} input signal of klystron, the effects of noise and/or drifts shall be mitigated.

As part of the development process, first the stability issue is analyzed and metrics for the stability are defined. Then the solution, a control system, is proposed. After that, the necessary building blocks of such a control system are treated in detail. From an evaluation of the sensors and actuators, the controller is designed and its positive effect on the gun stability is verified both by simulation and on the actual \gls{flute} accelerator. In both cases a considerable improvement is noticeable. 


\section*{Kurzfassung}

\chapter{Theoretical Framework}
\section{Linear Accelerators}
In a linear particle accelerator (LINAC) charged particles ,such as electrons, are accelerated to increase their total energy over their energy at rest.

Compared to heavier particles, such as protons ($m_p=\SI{938.27}{\mega\eV\per c\squared}$), electrons are fairly light ($m_p=\SI{0.511}{\mega\eV\per c\squared}$). Therefore they need to be brought to speeds comparable to the speed of light to achieve an useful energy increase. For this reason relativistic mechanics are needed to describe their movements.\footnote{As relativistic mechanics are a super set of classical mechanics, the equations also apply for slower particles.} \cite{Hinterberger1997}

With the speed of light $c=\SI{2.998e8}{\m\per\s}$ and the particle velocity $v$, it is common to define \cite{Wangler2008}:
\begin{align}\label{eq:theoreticalFramework_gammabeta}
\text{(normalized velocity)}\qquad\beta &= \frac{v}{c}\\
\text{(relativistic mass factor)}\qquad\gamma &= \frac{1}{\sqrt{1-\beta^2}}
\end{align}

The total energy of a particle is
\begin{equation}
U=\underbrace{\left(\gamma-1\right)mc^2}_{\text{kinetic engery } W} + \underbrace{mc^2}_{\text{rest energy}} = \gamma mc^2.
\end{equation}

With the kinetic electron energy out of the \gls{flute} electron gun of $W=\SI{7}{\mega\eV}$, rearranging \autoref{eq:theoreticalFramework_gammabeta} and using in $W=\left(\gamma-1\right)mc^2$ yield
\begin{align}
\gamma &= \frac{W}{m_e\,c^2}+1=\num{14.699}\\
\beta &= \sqrt{1-\nicefrac{1}{\gamma^2}}=\num{0.99768}.
\end{align}

\section{Signal Analysis}
\subsection{Auto correlation and Cross correlation}
The \textit{cross covariance} between two stochastic processes $x[n]$ and $y[n]$ is a measure of the similarity between $x[n]$ at index $n_1$ and $y[n]$ at index $n_2$ and is defined as
\begin{equation}\label{eq:crosscovariance}
r_{xy}[n_1,n_2] = \text{E}\left\{(x[n_1]-\mu_x[n_1])(y[n_2]-\mu_y[n_2])^\ast\right\}.
\end{equation}

For the special case of $y[n]:=x[n]$, $r_{xx}[n_1,n_2]$ is called \textit{auto covariance} and is a measure of self similarity of $x[n]$ \cite[p.~172]{Park2017}.

The processes $x[n]$ and $y[n]$ are called \textit{\gls{wss}} if the following two properties hold \cite[p.~167]{Park2017}. 
First, their means $\mu_{\xi}[n]$ are constant, i.e. they do not depend on the sample index:
\begin{align}
\mu_{x}[n] &= \mu_x\\
\mu_{y}[n] &= \mu_y
\end{align}
Also the auto covariance does not depend on the absolute sample indices $n_1$ and $n_2$, but merely on the difference between them:
\begin{equation}
r_{xy}[n_1,n_2] = r_{xy}[m],\qquad \text{with: } m:=n_2-n_1
\end{equation}

If both process in \autoref{eq:crosscovariance} are \gls{wss}, \autoref{eq:crosscovariance} simplifies to
\begin{equation}
r_{xy}[m] = \text{E}\left\{(x[n]-\mu_x)(y[n-m]-\mu_y)^\ast\right\}.
\end{equation}

For the auto covariance both means are identical and can be moved outside the expectation operator:
\begin{equation}
r_{xx}[m] = \text{E}\left\{(x[n])(y[n-m])^\ast\right\}-\mu_{x}^2.
\end{equation}

When analyzing signals, the stochastic processes are often unknown and only one realization $x[n]$ is known. But if the process generating $x[n]$ is \textit{(weakly) ergodic}, then one realization is enough to determine the auto covariance of the process \cite[p.~252]{Puente2019}.
Then the auto covariance can be estimated with
\begin{equation}\label{eq:autocovarianveEstimation}
\hat{r}[m] = \frac{1}{N} \sum_{n=m+1}^{N} x[n]\,x^\ast[n-m]\qquad m \in [0,\,N-1]
\end{equation} 


\subsection{Estimating the Spectrum of a Stochastic Process}
For a deterministic, time-discrete signal $x[n] \in \mathcal{L}_1$, the \gls{dft} exists\cite{Lapidoth2019} and is defined as
\begin{equation}
X[k] = \sum_{n=0}^{N-1} x[n]\,\text{e}^{-j\frac{2\pi}{N}k\,n}\qquad k,n \in [0,\,N-1],
\end{equation}
using $k=\frac{N}{2\pi}\,\omega = N\,f$ as the independent, discrete frequency variable. 
From the complex sequence $X[k]$, often only the magnitude (or energy) is of greater interest while the phase information are neglected. 
Therefore, $S_{xx}$ is defined as
\begin{equation}
S_{xx} = \left|X[k]\right|^2
\end{equation}
and called the \textit{\gls{esd}}.

If $x[n]$ is the realization of a stochastic process, then it is of random nature rather than deterministic.
Because realizations of physical processes do not posses finite energy, they are not in the $\mathcal{L}_1$ set and their \gls{dft} is not defined \cite[p.~5]{Stoica1997}.

In this case instead of an energy spectral density, the spectrum of the average power of the process, called the \textit{\gls{psd}}, can be used instead.
To compute the \gls{psd}, there are two possibilities:
\begin{align}
\Phi_{xx}[k] &= \sum_{m=-\infty}^{\infty} r[m]\,\text{e}^{-j\frac{2\pi}{N}k\,m} \label{eq:phixx1}\\
\Phi_{xx}[k] &= \lim_{N\rightarrow\infty} \text{E}\left\{\frac{1}{N}\left|\sum_{n=0}^{N-1} x[n]\,\text{e}^{-j\frac{2\pi}{N}k\,n}\right|^2 \right\} \label{eq:phixx2}
\end{align}
When assuming $r[m]$ decays ``fast enough'', i.e.
\begin{equation}
\lim_{N\rightarrow\infty} \frac{1}{N} \sum_{m=-N}^{N} |m|\,\left|r[m]\right| = 0
\end{equation}
then \autoref{eq:phixx1} and \autoref{eq:phixx2} are equal\cite[p.~7]{Stoica1997}.

For measured data however neither equations can be used directly.
For \autoref{eq:phixx1} the auto covariance sequence $r[m]$ is unknown.
But it could be estimated with \autoref{eq:autocovarianveEstimation}. In case of \autoref{eq:phixx2} it is not possible to evaluate the limit, because only finite length data can be sampled and also the expectation can not be computed since in general there is only one realization available. Both operations can be neglected when doing an estimation.

With these practical changes in place, \autoref{eq:phixx1} and \autoref{eq:phixx2} become
\begin{align}
\hat{\Phi}_{c,\,xx}[k] &= \sum_{m=-(N-1)}^{N-1} \hat{r}[m]\,\text{e}^{-j\frac{2\pi}{N}k\,m} \label{eq:phiCxx1}\qquad\text{(Correlogram)}\\
\hat{\Phi}_{p,\,xx}[k] &= \frac{1}{N} \left| \sum_{n=0}^{N-1} x[n]\,\text{e}^{-j\frac{2\pi}{N}k\,n}\right|^2\qquad\text{(Periodogram)}\label{eq:periodogram}.
\end{align}

Both methods yield equal results, if $r[m]$ is estimated with the biased estimator $\hat{r}[m]$ in \autoref{eq:autocovarianveEstimation} in contrast to the unbiased estimator (compare \cite[p.~24]{Stoica1997})
\begin{equation}
\hat{r}_{\text{unbiased}}[m] = \frac{1}{N-m} \sum_{n=m+1}^{N} x[n]\,x^\ast[n-m]\qquad m \in [0,\,N-1].
\end{equation}

\cite{Rowell2008} shows one key weakness of the unmodified periodogram method in \autoref{eq:periodogram}: The variance does not decrease significantly with more samples $N$. Instead the variance of the periodogram for each frequency approaches the square of the actual \gls{psd}:
\begin{equation}
\lim_{N\rightarrow\infty} \text{Var}\left\{\hat{\Phi}_{p,\,xx}[k]\right\} = \Phi_{xx}^2[k]
\end{equation}
Furthermore the periodogram/correlogram suffer from the smearing and leakage effects because the limited length of the data samples always causes an implicit windowing, thus reducing frequency resolution.\\

There are several popular methods that improve on the periodogram/correlogram concepts:

\textbf{Blackman-Tukey:} Because of the poor accuracy of $\hat{r}[m]$ for $k\approx N$ in the definition of $\hat{\Phi}_{c,\,xx}[k]$ and the bigger the $N$, the more small errors in $\hat{r}[m]$ sum up, truncating/windowing of $\hat{r}[m]$ with $w[k]$ (length $M$) can be beneficial for the accuracy of the estimation.
\begin{equation}
\hat{\Phi}_{BT,\,xx}[k] = \sum_{m=-(M-1)}^{M-1} w[k]\hat{r}[m]\,\text{e}^{-j\frac{2\pi}{N}k\,m}
\end{equation}
The choice of the window $w[k]$ trades frequency resolution for variance and smearing for leakage reduction \cite[p.~41]{Stoica1997}.

\textbf{Barlett:} The Barlett method reduces the variance of the periodogram by splitting the $N$ data samples in $Q=\nicefrac{N}{M}$ blocks and averaging together the sub-periodograms:
\begin{align}
\hat{\Phi}_{q,\,xx}[k] &= \frac{1}{M} \left| \sum_{n=0}^{M-1} x_q[n]\,\text{e}^{-j\frac{2\pi}{M}k\,n}\right|^2\\
\hat{\Phi}_{B,\,xx}[k] &= \frac{1}{Q} \sum_{q=1}^{Q} \hat{\Phi}_{q,\,xx}[k]
\end{align}
The variance of the estimation scales with $Q$ \cite[p.~6]{Rowell2008}:
\begin{equation}
\text{Var}\left\{\hat{\Phi}_{B,\,xx}[k]\right\} = \frac{1}{Q}\Phi_{xx}^2[k]
\end{equation}

\textbf{Welch:} The Welch method combines splitting the data into $Q$ segments with windowing each segment and allowing the segments to overlap. With $P = \nicefrac{1}{M} \sum_{n=0}^{M-1} |w[n]|^2$ being the ``power'' of the window, the Welch method is computed as
\begin{align}
\hat{\Phi}_{s,\,xx}[k] &= \frac{1}{M P} \left| \sum_{n=0}^{M-1} x_s[n]\,\text{e}^{-j\frac{2\pi}{M}k\,n}\right|^2\\
\hat{\Phi}_{W,\,xx}[k] &= \frac{1}{Q} \sum_{s=1}^{Q} \hat{\Phi}_{s,\,xx}[k].
\end{align}
Compared to the Barlett method, the overlapping of up to \SI{50}{\percent} (see \cite{Welch1967}) allows increasing $Q$, thus reducing the variance.
\begin{equation}\label{eq:varWelch}
\text{Var}\left\{\hat{\Phi}_{W,\,xx}[k]\right\} = \frac{1}{Q}\Phi_{xx}^2[k]
\end{equation}


In case of a non-stationary signal $x[n]$, one possibility to analyze and display the spectral content is the use of the \gls{stft} and the spectrogram, which is a two dimensional power spectral density function mapping frequency and time to a third coordinate like height, intensity or color.

To calculate the spectrogram, the signal is split into segments with the sliding window $w[n-m]$ for which duration the signal is assumed to be stationary. For each segment at time index $m$, the periodogram is calculated according to
\begin{equation}
\hat{\Phi}_{xx}[k,m] = \frac{1}{N} \left| \sum_{n=0}^{N-1} w[n-m] x[n]\,\text{e}^{-j\frac{2\pi}{N}k\,n}\right|^2.
\end{equation}









\section{Feedback Control Systems}
Feedback control systems are used to control a dynamic system (plant) in such a way that its output $y(t)$ follows a certain input $x(t)$ and disturbances on the output $d(t)$ are rejected. The general structure of a closed loop control system is shown in \autoref{fig:theoreticalFramework-feedback-architecture}. To achieve the tracking of the input and the stabilization of the output, a controller $G(s)$ uses the error $e(t)$ to control the plant $P(s)$ accordingly.
The error is defined as
\begin{equation}
e(t)=x(t)-r(t) = x(t)-[y(t)\ast h(t)]
\end{equation}
with $h(t)$ being the inverse Laplace transform of the filters transfer function $H(s)$.

\begin{figure}[tb]
	\centering
	\includegraphics[width=\textwidth]{chap/TheoreticalFramework/img/feedbackControl/architecture.tikz}
	\caption{General structure of a time continuous feedback control system}
	\label{fig:theoreticalFramework-feedback-architecture}
\end{figure}

Feedback control systems, or closed-loop systems, are to be differentiated from open-loop systems, in which there is no return path, so they cannot compensate for \textit{un}known disturbances. If $d(t)$ is known $\forall t$, then an open loop system would be possible and any errors could simply be compensated.

\subsection{Disturbance Rejection and Input Tracking}\label{sec:drandtrack}
Disturbance rejection and input tracking are two important characteristics to evaluate a stable controller. To calculate them, the block diagram in \autoref{fig:theoreticalFramework-feedback-architecture} and the Laplace transform of the inputs/outputs is used.\footnote{The Laplace transform of a function in time $f(t)$ is written as $F(s)=\mathcal{L}\left\{f(t)\right\}$.} 

To calculate how the output $y(t)$ depends on the input $x(t)$, the input tracking transfer function can be used.\cite[p.~88]{Foellinger2016} It is calculated as the transfer function $F_T=\nicefrac{Y(s)}{X(s)}$ by setting $d(t)=0$:
\begin{align}\label{eq:inputTracking}
Y(s)&=G(s)P(s)E(s)\quad\text{with:}\quad E(s)=X(s)-H(s)Y(s)\\
\Leftrightarrow Y(s)\left[1+G(s)P(s)H(s)\right]&=G(s)P(s)X(s)\\
\Leftrightarrow F_T:=\frac{Y(s)}{X(s)} &= \frac{G(s)P(s)}{1+G(s)P(s)H(s)}.
\end{align}

On the other hand, the transfer function $F_{DR}=\nicefrac{Y(s)}{D(s)}$ can be used to describe the systems response to a disturbance.\cite[p.~88]{Foellinger2016} It is defined by letting $x(t)=0$ and calculating
\begin{align}\label{eq:disturbanceRejection}
Y(s)&=G(s)P(s)E(s)+D(s)\quad\text{with:}\quad E(s)=-H(s)Y(s)\\
\Leftrightarrow Y(s)\left[1+G(s)P(s)H(s)\right]&=D(s)\\
\Leftrightarrow F_{DR}:=\frac{Y(s)}{D(s)}&=\frac{1}{1+G(s)P(s)H(s)}.
\end{align}

\subsection{Stability}
The application of a controller to a system is only useful if the resulting system has a stable behavior.
One possible definition of stability is the \gls{bibo} criterion\cite[p.~82]{Foellinger2016}:
\begin{definition}\label{def:bibo}
(\textit{BIBO stability}) A \gls{lti} system is said to be \gls{bibo} stable if for some $M,N \in \mathbb{R}^+$, the response to a bounded input $|u(t)| \le M$ results in a bounded output $|y(t)| \le N$.
\end{definition}

For a given control system, one way to analyze its stability is to plot the locus $z=F_o(s=j2\pi f)$ of the open loop frequency response
\begin{equation}
F_o(s) = G(s)P(s)H(s)
\end{equation}
from $f=0$ to $f=\infty$ and using the Nyquist stability criterion.
For the special case of a stable open loop $F_o(s)$ \footnote{The stability of $F_o(s)$ can often easily be determined from the block diagram.} the Nyquist stability criterion can be stated as\cite[p.~111]{Foellinger2016}
\begin{definition}\label{def:Nyquist}
(\textit{Nyquist stability criterion}) If the open loop $F_o(s)$ is stable, then the closed loop is stable if $z=F_o(s=j2\pi f)$ does not go through or encircles $z=-1$.
\end{definition}


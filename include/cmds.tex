%% -----------------------
%% |    Abbreviations    |
%% -----------------------
\newcommand{\op}[1]{\operatorname{#1}}                 % to write operators that
                                                       % are not predefined;
                                                       % it's just an abbrev.
                                                       % for the long command

\newcommand{\arr}[2]{\begin{array}{#1}#2\end{array}}   % to create arrays. very 
                                                       % useful in math env

\renewcommand{\d}{\ensuremath{\text{d}}}               % as the differential 
                                                       % operator, e.g. 
                                                       % \frac{\d x}{\d t}

\newcommand{\NN}{\mathbb{N}}                           % or change it to
\newcommand{\RR}{\mathbb{R}}                           % \mathbbm and include
\newcommand{\CC}{\mathbb{C}}                           % pkg bbm if you prefer

\newcommand{\pdb}[2]{\frac{\partial #1}{\partial #2}}  % partial derivative


%% -----------------------------------
%% |    Commands and Environments    |
%% -----------------------------------
\newcommand{\margtodo}                                 % used by \todo command
{\marginpar{\textbf{\textcolor{kitcolor}{ToDo}}}{}}
\newcommand{\todo}[1]
{{\textbf{\textcolor{kitcolor}{[\margtodo{}#1]}}}{}}   % for todo-notes inside 
                                                       % the document
\newenvironment{deprecated}                            % for something that you
{\begin{color}{gray}}{\end{color}}                     % want to use no more

\newcommand{\xcaption}[2]{\caption[#1]{\textbf{#1} #2}}% nice caption cmd for
                                                       % short and long descrip.

\newcommand{\xfigure}[5]{\begin{figure}[#1]            % a quick command for
\centering                                             % including graphics 
\includegraphics[scale=#2]{./fig/#3}                   % with all necessary vars
\xcaption{#4}{#5}
\label{fig:#3}
\end{figure}}

\newcommand{\xfigurerot}[5]{\begin{figure}[#1]        % same as above, only
\centering                                            % image is rotated
\includegraphics[angle=270,scale=#2]{./fig/#3}
\xcaption{#4}{#5}
\label{fig:#3}
\end{figure}}

\newcommand{\xtable}[4]{\begin{table}[#1]             % same for tables
\centering
\xcaption{#3}{#4}
\rowcolors{3}{gray!10}{white}
\include{./tab/#2}
\label{tab:#2}
\end{table}}



%% ------------------------------------
%% |    Quantum Mechanics and Math    |
%% ------------------------------------
\newcommand{\ket}[1]{\left|#1\right\rangle}           % \ket{X}  ->  |X>
\newcommand{\bra}[1]{\left\langle#1\right|}           % \bra{X}  ->  <X|
\newcommand{\braket}[2]                               % \braket{X}{Y}  ->  <X|Y>
{\left\langle#1 \middle| #2\right\rangle}
\newcommand{\bratenket}[3]                            % \bratenket{X}{Y}{Z}  ->
{\left\langle#1 \middle|\middle| #2 \middle|\middle|  % <X|Y|Z>
#3\right\rangle}
\newcommand{\anglemean}[1]                            % \anglemean{X}  ->  <X>
{\left\langle #1 \right\rangle}                       % \norm{X}  ->  || X ||
\newcommand{\norm}[1]{\left\lVert#1\right\rVert}

\newcommand{\updownarrows}                            % \ket\updownarrows  ->
{\text{\rotatebox[origin=c]{90}{$\rightleftarrows$}}} % |↑↓> (cmt is utf8!)
\newcommand{\downuparrows}                            % \ket\updownarrows  ->
{\text{\rotatebox[origin=c]{270}{$\rightleftarrows$}}}% |↓↑>
\newcommand{\neswarrows}                              % \ket\neswarrows  ->
{\text{\rotatebox[origin=c]{45}{$\rightleftarrows$}}} % |↗↙>
\newcommand{\swnearrows}                              % \ket\swnearrows  ->
{\text{\rotatebox[origin=c]{225}{$\rightleftarrows$}}}% |↙↗>

\newcommand{\cre}{c^\dagger}                          % annihalation operator
\newcommand{\anh}{c^{\vphantom{\dagger}}}             % creation operator
\newcommand{\numb}{n^{\vphantom{\dagger}}}            % number operator

\newcommand{\fullstop}{\text{\,.}}                    % fullstop or comma in
\newcommand{\comma}{\text{\,,}}                       % math mode for use
                                                      % after equations
                                                      
                                                      

\setcounter{secnumdepth}{3}           % Numbering also for \subsubsections
\setcounter{tocdepth}{3}              % Register \subsubsections in content dir

\setpapersize{A4}
\setmarginsrb{3cm}{1cm}{3cm}{1cm}     % {leftmargin}{topmargin}{rightmargin}...
             {6mm}{7mm}{5mm}{15mm}    % {bottommargin}{headheight}{headsep}...
                                      % {footheight}{footskip}

\setlength{\marginparwidth}{1.5cm}    % for todos to be positioned correctly

\parindent 0cm                        % do not indent beginning of paragraph
\parskip 1.5ex plus0.5ex minus0.5ex   % Margin between paragraphs





%% ------------------------
%% |    Language Setup    |
%% ------------------------
\newcommand{\SelectLanguage}[1]
{
    \AtBeginDocument
    {
        \selectlanguage{#1}           % babel command

        \iflanguage{ngerman}
        {
            \sisetup{output-decimal-marker={,}}
            % sets , for German and . otherwise
            \sisetup{list-final-separator={ und }}
            % "3, 4 and 5" in English or "3, 4 und 5" in German
            \sisetup{range-phrase={ bis }}
            % "1.5 to 1.8" in English or "1,5 bis 1,8" in German
            \sisetup{locale=DE}
            % e.g. using \cdot instead of \times for floating points
        }
        {
            \sisetup{output-decimal-marker=.}
            \sisetup{list-final-separator={ and }}
            \sisetup{range-phrase={ to }}
        }
    }
}





%% ---------------------------
%% |    Style of captions    |
%% ---------------------------
\newcommand{\changefont}[3]{\fontfamily{#1} \fontseries{#2}%
                            \fontshape{#3} \selectfont}
\newcommand{\chapterheadfont}{}

\renewcommand{\chaptername}{}

\makeatletter
\renewcommand{\section}{%
\@startsection{section}%
{1}                                    % Structure level
{0mm}                                  % Indention
{2ex plus 1ex minus 1ex}               % Pre-Margin
{0.5ex plus 0.5ex minus 0.5ex}         % Post-Margin
{\chapterheadfont\large\bfseries}      % Style
}
\renewcommand{\subsection}{%
\@startsection{subsection}%
{2}                                    % Structure level
{0mm}                                  % Indention
{1.5ex plus 1ex minus 0.5ex}           % Pre-Margin
{0.3ex plus 0.3ex minus 0.3ex}         % Post-Margin
{\chapterheadfont\large\bfseries}      % Style
}
\renewcommand{\subsubsection}{%
\@startsection{subsubsection}%
{3}                                    % Structure level
{0mm}                                  % Indention
{1.5ex plus 1ex minus 0.5ex}           % Pre-Margin
{0.2ex plus 0.2ex minus 0.2ex}         % Post-Margin
{\chapterheadfont\normalsize\bfseries} % Style
}
\renewcommand{\paragraph}{%
\@startsection{paragraph}%
{4}                                    % Structure level
{0mm}                                  % Indention
{1.3ex plus 1ex minus 0.3ex}           % Pre-Margin
{0.2ex plus 0.2ex minus 0.2ex}         % Post-Margin
{\chapterheadfont\normalsize\bfseries} % Style
}
\renewcommand{\subparagraph}{%
\@startsection{subparagraph}%
{5}                                    % Structure level
{0mm}                                  % Indention
{1ex plus 1ex minus 0.2ex}             % Pre-Margin
{0.1ex plus 0.1ex minus 0.1ex}         % Post-Margin
{\chapterheadfont\normalsize\bfseries} % Style
}
\makeatother




%% -----------------------------------
%% |    Style of chapter captions    |
%% -----------------------------------
\newlength{\chapnolen}
\newlength{\chapparlen}
\newsavebox{\chapno}
\makeatletter
\renewcommand{\@makechapterhead}[1]{
    \vspace*{0.1\textheight}
    \vskip 15\p@
    {\parindent \z@ \raggedright \normalfont
        \ifnum \c@secnumdepth >\m@ne
            \if@mainmatter
                \savebox{\chapno}{\chapterheadfont\huge\bfseries \thechapter.}
                \settowidth{\chapnolen}{\usebox{\chapno}}
                \parbox[t]{\chapnolen}{\usebox{\chapno}}\nobreak\leavevmode
            \fi
        \fi
        \interlinepenalty\@MM
        \setlength{\chapparlen}{\textwidth}
        \addtolength{\chapparlen}{-1.0\chapnolen}
        \addtolength{\chapparlen}{-2ex}
        \leavevmode\nobreak
        \parbox[t]{\chapparlen}%
        {\raggedright\chapterheadfont\huge \bfseries #1\par\nobreak}
        \vskip 30\p@
    }}


\renewcommand{\@makeschapterhead}[1]{
    \vspace*{50\p@}
    {\parindent \z@ \raggedright
        \normalfont
        \interlinepenalty\@M
        \chapterheadfont \huge \bfseries  #1\par\nobreak
        \vskip 40\p@
    }
}





%% ------------------------------------
%% |    Style of content directory    |
%% ------------------------------------
\let\oldtableofcontents\tableofcontents
\renewcommand{\tableofcontents}{{\pdfbookmark{\contentsname}{\contentsname}%
\chapterheadfont\oldtableofcontents}}
\let\@olddottedtocline\@dottedtocline
\renewcommand{\@dottedtocline}[5]{\@olddottedtocline{#1}{#2}{#3}{#4}%
{\chapterheadfont #5}}
\makeatother




%% ------------------------------------------
%% |    Style of appendix and mainmatter    |
%% ------------------------------------------
\newcommand{\FrontMatter}
{
    \frontmatter

    \pagestyle{empty}

    \fancypagestyle{plain}{            % to ensure toc page style is really 
                                       % empty (it uses \thispagestyle{plain})
        \fancyhf{}                     % clear all header and footer fields
        \fancyfoot{}                   % except the center
        \renewcommand{\headrulewidth}{0pt}
        \renewcommand{\footrulewidth}{0pt}
    }
}

\newcommand{\MainMatter}
{
    \clearpage

    \begingroup                        % make sure that there is no involuntary 
                                       % blankpage added after toc.
    \let\cleardoubleoddstandardpage\relax
    \mainmatter
    \endgroup

    \frontmatter \pagestyle{empty}

    \fancypagestyle{plain}{            % redefine chapter first page style,
                                       % which is redefined by \FrontMatter
        \fancyhf{}                     % clear all header and footer fields
        \fancyfoot[C]{\thepage}        % except the center
        \renewcommand{\headrulewidth}{0pt}
        \renewcommand{\footrulewidth}{0pt}
    }

    \mainmatter
    \pagestyle{fancy}
    \renewcommand{\chaptermark}[1]{\markboth{\chaptername\ %
                                             \thechapter.\ ##1}{}}
    \lhead[\thepage]{\leftmark}\chead[]{}\rhead[\thesispagehead]{\thepage}
    \lfoot{}\cfoot{}\rfoot{}
}

\newcommand{\Appendix}
{
    \clearpage
    \appendix
    \setcounter{section}{0}
    \setcounter{subsection}{0}
    \setcounter{figure}{0}
    \setcounter{equation}{0}
    \renewcommand\thesection{\Alph{section}}
    \renewcommand\thefigure{\Alph{section}.\arabic{figure}}
    \renewcommand\thetable{\Alph{section}.\arabic{table}}
    \renewcommand\theequation{\Alph{section}.\arabic{equation}}
    \numberwithin{equation}{section}
    \lhead[\thepage]{Appendix}
}

\newcommand{\TheBibliography}
{
    \clearpage
    \thispagestyle{plain}
}

\newcommand{\emptychapter}[2][]
{
    \addtocounter{chapter}{1}
    \addtocontents{toc}{\protect\contentsline
        {chapter}{\protect\numberline {\thechapter}#2}{#1}{}}
}


%mystuff
%floatbarriers also for subsections
%https://tex.stackexchange.com/questions/118662/use-placeins-for-subsections
\makeatletter
\AtBeginDocument{%
  \expandafter\renewcommand\expandafter\subsection\expandafter{%
    \expandafter\@fb@secFB\subsection
  }%
}
\makeatother

\makeatletter
\AtBeginDocument{%
  \expandafter\renewcommand\expandafter\subsubsection\expandafter{%
    \expandafter\@fb@secFB\subsubsection
  }%
}
\makeatother
